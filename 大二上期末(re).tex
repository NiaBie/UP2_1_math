\documentclass[11pt, a4paper, UTF8]{ctexart}
\usepackage{amsmath}
\usepackage{bm}
\usepackage{multirow}
\usepackage{esint}
\usepackage{geometry}
\usepackage{amssymb}
\pagestyle{empty}
\definecolor{brightpink}{rgb}{1.0, 0.0, 0.5}
\definecolor{ao}{rgb}{0.0, 0.5, 0.0}
\usepackage[colorlinks,linkcolor=ao,anchorcolor=brightpink,citecolor=green,CJKbookmarks=True]{hyperref}
\CTEXsetup[format={\Large\bfseries}]{section}
\geometry{a4paper,top=1cm,bottom=1cm,left=2cm}
%\renewcommand{\baselinestretch}{1}\normalsize
\begin{document}
\indent\\
\large
\raggedright
\def\fuck{\maltese}
\def\shit#1{#1\protect\hyperlink{catalog}{$\fuck$}}
\hypertarget{catalog}{}
\tableofcontents
\setcounter{secnumdepth}{-1}
\section{\shit{已有套路}}
(1)$A,B$为正交矩阵,$(A+B)^T=A^T(A+B)B^T$\\
(2)实对称阵的特征值\\
(3)实对称阵特征向量正交\\
(4)行列式的任一行(列)元素与另一行(列)元素的代数余子式乘积之和等于0\\
(5)线性空间8性质\\
(6)对角化$P^{-1}AP=D$\\
(7)构造特征向量\\
(8)$Ax=0$解向量为$\alpha_1,\alpha_2,\begin{bmatrix}
\alpha_1^T\\
\alpha_2^T
\end{bmatrix}x=0$的解向量为$\beta_1,\beta_2$,$\begin{bmatrix}
\beta_1^T\\
\beta_2^T\\
\end{bmatrix}x=0$的解向量为$\alpha_1,\alpha_2$\\
(9)$(Bx)^TA(Bx)=x^T(B^TAB)x$\\
(10)坐标/矩阵\\
向量的坐标$\alpha=\varepsilon X$\\
$\varepsilon$到$\eta$的过渡矩阵为$T,\alpha=\varepsilon X=\eta Y,X=TY$\\
线性变换$\bm A$在$\varepsilon$下的矩阵为$A:\bm A\varepsilon=\varepsilon A$\\
线性变换$\bm A$在$\varepsilon$下的矩阵为$A,\alpha=\varepsilon X,\bm A\alpha=AX$\\
$\varepsilon$到$\eta$的过渡矩阵为$T,\bm A\varepsilon=\varepsilon A,\bm A\eta=\eta B,B=T^{-1}AT$\\
(11)$A$与$\varLambda$相似,$|A-\lambda E|=|\varLambda-\lambda E|$\\
(12)$r((N^TM)^3)\le r(N^TMN^TMN^T)$\\
(13)高斯消元,增广矩阵的秩\\
(14)齐次方程组和非齐次方程组解的结构\\
(15)$N$为基础解系,$x$为基础解系组合,$\exists y,x=Ny$,(看成是系数?)\\
(16)$A^{n+1}\alpha=0,A^n\alpha\not=0\to\alpha,A\alpha,...,A^n\alpha$线性无关\\
(令$k_0\alpha+k_1A\alpha+...+k_nA^n\alpha=0\Rightarrow A^n(k_0\alpha+k_1A\alpha+...+k_nA^n\alpha)=0$)\\
(17)相似矩阵有相同特征值\\
(18)若$P^{-1}AP=B,A$的特征向量为$\alpha$,$B$的特征向量为$\beta$,则$\alpha=P\beta$
\[B(P^{-1}\xi)=(P^{-1}AP)P^{-1}\xi=P^{-1}A\xi=P^{-1}\lambda\xi=\lambda P^{-1}\xi\]
(19)范德蒙行列式\\
\[\begin{vmatrix}
1&1&1&1\\
x_1&x_2&...&x_n\\
x_1^2&x_2^2&...&x_2^n\\
...&...&...&...\\
x_1^{n-1}&x_2^{n-1}&...&x_n^{n-1}\\
\end{vmatrix}\]
(20)
\[\begin{vmatrix}
1&&&1\\
1&1&\\
&...&...&\\
&&1&1
\end{vmatrix}=1+(-1)^{n+1}\]
\section{\shit{其他}}
(1)施密特正交化\\
\end{document}