\documentclass[11pt, a4paper, UTF8]{ctexart}
\usepackage{amsmath}
\usepackage{bm}
\usepackage{multirow}
\usepackage{esint}
\usepackage{geometry}
\usepackage{amssymb}
\pagestyle{empty}
\definecolor{brightpink}{rgb}{1.0, 0.0, 0.5}
\definecolor{ao}{rgb}{0.0, 0.5, 0.0}
\usepackage[colorlinks,linkcolor=ao,anchorcolor=brightpink,citecolor=green,CJKbookmarks=True]{hyperref}
\CTEXsetup[format={\Large\bfseries}]{section}
\geometry{a4paper,top=1cm,bottom=1cm,left=2cm}
%\renewcommand{\baselinestretch}{1}\normalsize
\begin{document}
\indent\\
\large
\raggedright
\def\fuck{\maltese}
\def\shit#1{#1\protect\hyperlink{catalog}{$\fuck$}}
\hypertarget{catalog}{}
\tableofcontents
\setcounter{secnumdepth}{-1}
\section{\shit{预备知识}}
\subsection{\shit{线性方程组}}
\subsubsection{\shit{系数矩阵和增广矩阵}}
\[
A=\begin{bmatrix}
a_{11}&a_{12}&...&a_{1n}\\
a_{21}&a_{22}&...&a_{2n}\\
&&...&\\
a_{m1}&a_{m2}&...&a_{mn}\\
\end{bmatrix}
\qquad\bar A=\begin{bmatrix}
a_{11}&a_{12}&...&a_{1n}&b_1\\
a_{21}&a_{22}&...&a_{2n}&b_2\\
&&&...&\\
a_{m1}&a_{m2}&...&a_{mn}&b_m\\
\end{bmatrix}\]
\subsubsection{\shit{$Ax=b$有解的充要条件}}
\[r(A)=r(\bar A)\]
在此条件下:\\
1.若$r(A)=n$,则方程组$Ax=b$有唯一解\\
2.若$r(A)<n$,则方程组$Ax=b$有无穷多组解
\subsubsection{\shit{行列式与线性方程组}}
$A_{n\times n}x=0$有非零解的充要条件是系数矩阵的行列式$|A|=0$\\
$A_{n\times n}x=b$有唯一解的充要条件是$|A|\not=0$,唯一解为$x=A^{-1}b$
\subsection{\shit{基变换/过渡矩阵}}
\[(\eta_1.\eta_2...\eta_n)=(\varepsilon_1,\varepsilon_2...\varepsilon_n)\begin{bmatrix}
t_{11}&t_{12}&...&t_{1n}\\
t_{21}&t_{22}&...&t_{2n}\\
&&...&\\
t_{n1}&t_{n2}&...&t_{nn}\\
\end{bmatrix}=(\varepsilon_1,\varepsilon_2...\varepsilon_n)T\]
称$T$为从基$(\varepsilon_1,\varepsilon_2...\varepsilon_n)$到基$(\eta_1.\eta_2...\eta_n)$的过渡矩阵\\
\subsubsection{\shit{命题}}
给定一组基$\varepsilon_1,\varepsilon_2...\varepsilon_n$,命
\[(\eta_1.\eta_2...\eta_n)=(\varepsilon_1,\varepsilon_2...\varepsilon_n)T\]
1.如果$\eta_1.\eta_2...\eta_n$是一组基,则$T$可逆\\
2.如果$T$可逆,则$\eta_1.\eta_2...\eta_n$是一组基
\subsection{\shit{坐标变换}}
设向量$\alpha$在$\varepsilon_1,\varepsilon_2...\varepsilon_n$下的坐标为$(x_1,x_2...x_n)$,即
\[\alpha=(\varepsilon_1,\varepsilon_2...\varepsilon_n)\begin{bmatrix}
x_1\\
x_2\\
...\\
x_n\\
\end{bmatrix}=(\varepsilon_1,\varepsilon_2...\varepsilon_n)X\]
在$\eta_1.\eta_2...\eta_n$下的坐标为$(y_1,y_2...y_n)$,即\\
\[\alpha=(\eta_1.\eta_2...\eta_n)\begin{bmatrix}
y_1\\
y_2\\
...\\
y_n\\
\end{bmatrix}=(\eta_1.\eta_2...\eta_n)Y\]
设$(\varepsilon_1,\varepsilon_2...\varepsilon_n)$到$(\eta_1.\eta_2...\eta_n)$的过渡矩阵为$T$,坐标变换公式:
\[X=TY\]
\subsection{\shit{坐标变换公式}}
向量$\gamma$在基底$\alpha_1,\alpha_2...\alpha_n$的坐标是$x_1,x_2...x_n$,在基底$\beta_1,\beta_2...\beta_n$下的坐标是$y_1,y_2...y_n$,则坐标变换公式是:\\
\[\begin{bmatrix}
x_1\\
x_2\\
...\\
x_n
\end{bmatrix}=C
\begin{bmatrix}
y_1\\
y_2\\
...\\
y_n\\
\end{bmatrix}\]
\subsection{\shit{线性变换}}
设$U,V$是$K$上的线性空间,$T:U\to V$,若:\\
\[T(\lambda x+\mu y)=\lambda T(x)+\mu T(y)\]
则称$T$为线性空间$U\to V$上的一个线性变换.从线性空间$U$到其自身的线性变换称为$U$上的线性变换\\
\subsubsection{\shit{线性变换的矩阵表示}}
记$\{a_i\},\{b_i\}$为$U$和$V$的一组基,$T:U\to V$.设$U$中向量$x$用基$\{a_i\}$表示为:\\
\[x=\alpha_1a_1+\alpha_2a_2+...+\alpha_na_n\]
由线性变换性质得:\\
\[T(x)=\alpha_1T(a_1)+\alpha_2T(a_2)+...+\alpha_nT(a_n)\]
又$T(a_i)\in V$,所以可以用基$\{b_i\}$表示:\\
\[T(a_i)=[b_1,b_2...b_n]\begin{bmatrix}
c_{1i}\\
c_{2i}\\
...\\
c_{ni}
\end{bmatrix}\]
记:\\
\[C=\begin{bmatrix}
c_{11}&c_{12}&...&c_{1n}\\
c_{21}&c_{22}&...&c_{2n}\\
&&...\\
c_{n1}&c_{n2}&...&c_{nn}\\
\end{bmatrix}\]
则:\\
\[[T(a_1),T(a_2)...T(a_n)]=[b_1,b_2...b_n]C\]
所以:\\
\[T(x)=\alpha_1T(a_1)+\alpha_2T(a_2)+...+\alpha_nT(a_n)=[T(a_1),T(a_2)...T(a_n)]\begin{bmatrix}
\alpha_1\\
\alpha_2\\
...\\
\alpha_n
\end{bmatrix}
=[b_1,b_2...b_n]C
\begin{bmatrix}
\alpha_1\\
\alpha_2\\
...\\
\alpha_n
\end{bmatrix}\]
称$C$为线性变换$T$在基$\{a_i\}$下的矩阵\\
\section{欧氏空间\protect\hyperlink{catalog}{$\fuck$}}
\subsection{\shit{内积}}
设$(V,R,+,\cdot)$是一个向量空间\\
若映射$\langle\cdot,\cdot\rangle:V\times V\to\mathbb{R}$满足:\\
1.$\langle x,y\rangle=\langle y,x\rangle$,对称性\\
2.$\langle x,x\rangle\ge0,=0\Leftrightarrow x=0$\\
3.$\langle\alpha x+\beta y,z\rangle=\alpha\langle x,z\rangle+\beta\langle y,z\rangle,\forall\alpha,\beta\in\mathbb{R},\forall x,y,z\in V$\\
则称$\langle x,y\rangle$是$x$与$y$的内积,$\langle\cdot,\cdot\rangle$称为$V$上的一个内积\\
$(V,\mathbb{R},+,\cdot,\langle,\rangle)$称为欧氏空间\\
例\\
$\mathbb{R}=\{(x_1,x_2\cdots x_n)|x_j\in\mathbb{R},j=1,2\cdots n\}$\\
(1)点积: $x\cdot y=\sum\limits_{j=1}^nx_jy_j$\\
(2)设$\lambda_1,\lambda_2\cdots\lambda_n>0$,令$\langle x,y\rangle_\lambda=\sum\limits_{j=1}^n\lambda_jx_jy_j$\\
(3)设矩阵$A$满足:\\
(3.1)$A^T=A$,对称\\
(3.2)$\forall x\in\mathbb{R},xAx^T\ge0,=0\Leftrightarrow x=0$,(正定对称矩阵),$\sum\limits_{ij=1}^nx_ia_{ij}x_j\ge0?$\\
对$x,y\in V$,定义$\langle x,y\rangle_A=xAy^T$,则$\langle\cdot,\cdot\rangle_A$是$\mathbb{R}^n$上一个内积\\
$C([0,1]),R^2([0,1])\quad\langle f,g\rangle=\int_0^1f(t)g(t)dt$\\
\subsection{模(长)(范数)}
$|x|(||x||?)=\sqrt{\langle x,x\rangle}$\\
距离:$d(x,y)=|x-y|$
模的性质:\\
(1)$|x|\ge0,=0\Leftrightarrow x=0$\\
(2)$|\alpha x|=|\alpha||x|$\\
(3)三角不等式,$|x\pm y|\le|x|+|y|$\\
证:\\
$\langle x+ty,x+ty\rangle,t\in\mathbb{R}=\langle x,x\rangle+2t\langle x,y\rangle+t^2\langle y,y\rangle=|x|^2+2t\langle x,y\rangle+t^2|y^2|$,$\forall t\in\mathbb{R}$\\
$(2\langle x,y\rangle)^2\le|x||y|$,(...不等式)\\
$|x+y|^2=\langle x+y,x+y\rangle=|x|^2+|y|^2+2\langle x,y\rangle$
$\le|x|^2+|y|^2+2|x||y|$\\
夹角:$\cos L(x,y)=\dfrac{\langle x,y\rangle}{|x||y|}$\\
\subsection{正交}
若$\langle x,y\rangle=0$,称$x$与$y$正交,任一向量与零向量正交\\
记为$x\perp y$\\
勾股定理:若$x\perp y$,则$|x+y|^2=|x|^2+|y|^2$\\
$|x+y|^2=\langle x+y,x+y\rangle=\langle x,x\rangle+\langle y,y\rangle+2\langle x,y\rangle=|x|^2+|y|^2$\\
$V,n$维欧氏空间,向量$z_1,z_2\cdots z_n$基\\
$x=x_1z_1+x_2z_2+\cdots+x_nz_n$\\
$y=y_1z_1+y_2z_2+\cdots+y_nz_n$\\
$\langle x,y\rangle=\langle\sum\limits_{i=1}^nx_iz_i,\sum\limits_{j=1}^ny_jz_j\rangle$\\
$=\sum\limits_{ij=1}^nx_ia_{ij}y_j=XAY^T,X=(x_1,x_2\cdots x_n)$\\
其中$A=(\langle z_i,z_j\rangle)_{n\times n}$是一个正交对称阵\\
称为$(V,\langle,\rangle)$的度量矩阵\\
现设$w_1,w_2\cdots w_n$是一组基\\
$(w_1,w_2\cdots w_n)=(z_1,z_2\cdots z_n)C$\\
$w_1=z_1c_{11}+z_2c_{21}+\cdots+z_nc_{n1}$(C是过渡矩阵)\\
记$A$为$(z_1,z_2\cdots z_n)$下的度量阵??\\
$B$为$(w_1,w_2\cdots w_n)$...\\
$b_{ij}=\langle w_i,w_j\rangle=\langle\sum\limits_kz_kc_{ki},\sum\limits_lz_lc_{lj}\rangle$\\
$=\sum\limits_{k,l}c_{ki}\langle z_k,z_l\rangle c_{lj}$\\
$=\sum\limits_{k,l}c_{ki}a_{kl}c_{lj}$\\
所以$B=C^TAC$\\
合同:\\
若存在可逆阵$C$使得$B=C^TAC,A=(C^{-1})^TBC^{-1}$\\
$A=(\langle\bm z_i,\bm z_j\rangle)_{n\times n}$\\
$\bm x=\sum x_i\bm z_i,\bm y=\sum y_j\bm z_j$\\
$\langle\bm x,\bm y\rangle=X^TAY$\\``''
$A^T=A$且正定\\
$C^TAC=B,C^T(\bm z_j)C=(\bm w_j)$\\
$(\bm w_1,\bm w_2\cdots\bm w_n)=(\bm z_1,\bm z_2,\cdots\bm z_n)C$\\
\section{正交\protect\hyperlink{catalog}{$\fuck$}}
\subsection{标准正交基\protect\hyperlink{catalog}{$\fuck$}}
$\bm x_1,\bm x_2,\cdots\bm x_n\in V_n$\\
若两两正交,则称$\{\bm x_1,\bm x_2,\cdots\bm x_n\}$为正交组\\
\subsubsection{\shit{标准正交基}}
若$\{\bm x_1,\bm x_2,\cdots\bm x_n\}$是正交组,则$\{\bm x_1,\bm x_2,\cdots\bm x_n\}$是线性无关组,故$r\le n$\\
$n$个向量的正交组称为$V_n$的一组正交基\\
若每个向量的模都是1:标准正交基\\
例\\
$\mathbb{R}^3,\bm e_1=(1,0,0),\bm e_2=(0,1,0),\bm e_3=(0,0,1)$\\
i.e.$\{\bm e_1,\bm e_2,\cdots\bm e_n\}$满足$\langle\bm e_i,\bm e_j\rangle=S_{ij}$\\
i.e.其度量矩阵为$I_n$\\
$\bm x=(x_1,x_2,x_3)=\sum_{i=1}^3x_i\bm e_i$\\
$\bm y=(y_1,y_2,y_3)=\sum_{j=1}^3y_j\bm e_j$\\
$\langle\bm x,\bm y\rangle=\langle\sum_{i=1}^nx_i\bm e_i,\sum_{j=1}^ny_j\bm e_j\rangle$\\
$\bm x=\sum_{i=1}^nx_i\bm e_i$\\
$\langle\bm x_i,\bm e_j\rangle=\langle\sum_{j=1}^nx_i\bm e_i,\bm e_i\rangle$\\
$x_j=\langle\bm x,\bm e_j\rangle$\\
$\bm x=\sum_{j=1}^n\langle x,\bm e_j\rangle\bm e_j$\\
$|\bm x|^2=\langle x_i\bm e_i,\sum x_j\bm e_j\rangle=\sum_{j=1}^nx_j^2$\\
$\int_Tf^2(x)dx=|\langle f,\bm e_0\rangle|^2+|\langle f,\bm e_1\rangle|^2+|\langle f,\bm e_2\rangle|^2+\cdots$\\
$f\to\{\langle c_j,e_j\rangle\}_{j=0}^\infty$\\
$\int_\Pi f^2(t)dt=\sum_{j=0}^\infty c^2_j$\\
$l^2=\{\{a_n\}_{n=0}^\infty|\sum_{n=0}^\infty a_n^2<\infty\}$\\
$\bm x_2-c\bm x_1\perp\bm x_1$\\
$\langle\bm x_2-c\bm x_1,\bm x_1\rangle=0$\\
所以$c=\dfrac{\langle\bm x_1,\bm x_2\rangle}{\langle\bm x_1,\bm x_1\rangle}=\dfrac{\langle\bm x_1,\bm x_2\rangle}{|\bm x_1|^2}$\\
给定$\bm x_1\bm x_2$,称$\dfrac{\langle x_1,\bm x_2\rangle}{|\bm x_1|^2}\bm x_1=Proj_{\bm x_1}\bm x_2$为$\bm x_2$在$\bm x_1$上的正交投影,这是因为\\
$\bm x_2-\dfrac{\langle\bm x_1,\bm x_2\rangle}{|\bm x_1|^2}\bm x_1\perp\bm x_1$,而span$\{\bm x_1,\bm x_2\}=span\{\bm x_1,\bm x_2-\dfrac{\langle\bm x_1,\bm x_2\rangle}{|\bm x_1|^2}\bm x_1\}$\\
$\bm x_1,\bm x_2,\bm x_3$线性无关\\
$\bm y_1,\bm y_2,\bm y_3$\\
$\bm x_1,\bm x_2-\dfrac{\langle\bm x_1,\bm x_2\rangle}{|\bm x_1|^2}\bm x_1,\bm x_3-\dfrac{\langle\bm x_3,\bm y_1\rangle}{|\bm y_1|^2}\bm y_1-\dfrac{\langle\bm x_3,\bm y_2\rangle}{|\bm y_2|^2}\bm y_2$\\
$\bm y_3=\bm x_3-\langle\bm x_3,\bm e_1\rangle\bm e_1-\langle\bm x_3,\bm e_2\rangle\bm e_2$\\
$\bm e_1=\dfrac{\bm x_1}{|\bm x_1|},\bm e_2=\dfrac{\bm y_2}{|\bm y_2|}$\\
$\bm y_4=\bm x_4-\sum_{j=1}^3\dfrac{\langle\bm x_4,\bm y_j\rangle}{|\bm y_j|^2}\bm y_j=\bm x_4-\sum_{j=1}^3\langle\bm x_4,\bm e_j\rangle\bm e_j$\\
\subsection{\shit{Gram-Schmidt正交化方法}}
给定一组线性无关向量$\{\bm x_1,\bm x_2,\cdots\bm x_r\}$\\
令$\bm y_1=\bm x_1,\bm y_2=\bm x_2-\dfrac{\langle\bm x_2,\bm y_1\rangle}{\langle\bm y_1,\bm y_1\rangle}\bm y_1,\bm y_r=\bm x_r-\sum_{j=1}^r\dfrac{\langle\bm x_r,\bm y_j\rangle}{\langle\bm y_j,\bm y_j\rangle}\bm y_j$\\
例
$\mathbb{R}^4,\bm x_1=(1,1,0,0),\bm x_2=(1,0,1,0),\bm x_3=(1,0,0,-1)$\\
用Gran-Schmidt正交化方法化为标准正交系\\
$\langle\bm x_1\bm x_1\rangle=2,\langle\bm x_2,\bm x_1\rangle=1$\\
所以$\bm y_1=\bm x_1,\bm y_2=\bm x_2-\bm x_2-\dfrac{\langle\bm x_2,\bm x_1\rangle}{\langle\bm x_1,\bm x_1\rangle}\bm x_1=(1,0,1,0)-(\dfrac{1}{2},\dfrac{1}{2},0,0)$\\
$\langle\bm y_2,\bm y_2\rangle=\dfrac{3}{2}$\\
$\bm y_3=(\dfrac{1}{3},-\dfrac{1}{3},-\dfrac{1}{3},-1)$\\
$\langle\bm y_3,\bm y_3\rangle=\dfrac{4}{3}$\\
$\bm e_1=\dfrac{\bm x_1}{|\bm x_1|}=(\dfrac{\sqrt{2}}{2},\dfrac{\sqrt{2}}{2},0,0),\bm e_2=\dfrac{\bm y_2}{|\bm y_2|}=(\dfrac{\sqrt{6}}{6},-\dfrac{\sqrt{6}}{6},\dfrac{\sqrt{6}}{3},0),\bm e_3=(\dfrac{\sqrt{3}}{6},-\dfrac{\sqrt{3}}{6},-\dfrac{\sqrt{3}}{6},-\dfrac{\sqrt{3}}{2})$\\
$\bm x,\{\bm e_1,\bm e_2\cdots\bm e_n\}$\\
$\bm x=\langle\bm x,\bm e_1\rangle\bm e_1+\langle\bm x,\bm e_2\rangle\bm e_2+\cdots+\langle\bm x,\bm e_n\rangle\bm e_n$\\
\subsection{\shit{正交矩阵}}
设$\bm e_1,\bm e_2\cdots\bm e_n$和$\bm y_1,\bm y_2\cdots\bm y_n$是$V$中两组标准正交基\\
$(\bm y_1,\bm y_2\cdots\bm y_n)=(\bm e_1,\bm e_2\cdots\bm e_n)U$\\
$U=(u_{ij})_{n\times n}$\\
$=(\sum_{j=1}u_{1j}\bm e_j,\sum u_{2j}\bm e_j,\cdots,\sum u_{nj}\bm e_j)$\\
$S_{ij}=\langle\bm\eta_i,\bm \eta_j\rangle=U^TI_nU=(U^TU)_{ij}$\\
所以$U^TU=I_n$\\
\subsubsection{\shit{正交矩阵}}
若$n$阶方阵$U$满足\\
\[U^TU=I_n\]\\
则称$U$为一个正交(方)阵,即$U^{-1}=U^T$\\
\subsection{\shit{正交变换}}
设$\mathbb{U}:V_n\to V_n$线性变换,若$\forall\bm x,\bm y\in V_n$\\
\[\langle\mathbb{U}\bm x,\mathbb{U}\bm y\rangle=\langle\bm x,\bm y\rangle\]则称$\mathbb{U}$是$V_n$上的一个正交变换\\
$u_{ik}x_ku_{il}y_l=\bm x^TU^TU\bm y=\bm x^T\bm y\mbox{这是什么??}$\\
\section{\shit{复习}}
\subsection{\shit{正交}}
$\bm e_1,\bm e_2...\bm e_n$(为正交基??),$\{e_j\}_{j=1}^n$的度量矩阵是单位阵\\
\[(\bm\eta_1,\bm\eta_2...\bm\eta_n)=(\bm e_1,\bm e_2...\bm e_n)U\]
$UU^T=E$即$U^{-1}=U^T$\\
\[1=|E|=|U^TU|=|U^T||U|=|U|^2\]
所以$|U|=\pm1$(列???)\\
在$\mathbb{R}^n$中,一组标准正交基$\{\bm\eta_1,\bm\eta_2...\bm\eta_n\}$构成的矩阵$U=(\bm\eta_1,\bm\eta_2...\bm\eta_n)$是一个正交阵:\\
\[U^TU=E\]\\
(???)$\langle\bm a,\bm b\rangle=\sum_{j=1}^na_jb_j=\bm a^T\bm b$\\
\subsection{\shit{正交变换}}
例:\\
$U:\mathbb{R}^2\to\mathbb{R}^2$,旋转变换:\\
\[\begin{bmatrix}
\cos\theta&-\sin\theta\\
\sin\theta&\cos\theta
\end{bmatrix}\]
$U^TU=\begin{bmatrix}
1&0\\
0&1
\end{bmatrix}$
\section{\shit{特征值}}
\[\begin{bmatrix}
\lambda-\cos\theta&\sin\theta\\
-\sin\theta&\lambda-\cos\theta
\end{bmatrix}=(\lambda-\cos\theta)^2+\sin\theta^2=\lambda^2-2\lambda+1\]
\subsection{\shit{正交阵的特征值模为1}}
\[U\bm\xi=\lambda\bm\xi\]
取共轭转置:\\
\[\bar{\bm\xi}^T\bar U^T=\bar\lambda\bar{\bm\xi}^T,(\bar U^T=U^T)\]
\[\bar{\bm\xi}U^TU\bm\xi=|\lambda|^2\bar{\bm\xi}^T\bm\xi\]
即\\
\[|\bm\xi|^2=|\lambda|^2|\bm\xi|^2\]
但$\bm\xi\not=\bm0$,所以$|\lambda|^2=1$\\
\section{\shit{$(V_n,\mathbb{R},+,\cdot)$同构于$(\mathbb{R}^n,\mathbb{R},+,\cdot)$}}
\[\sigma:\bm z=\sum_{j=1}^nx_j\bm z_j(\bm z\in V\to\bm x=(x_1,x_2...x_n)\in\mathbb{R}^n)\]
$\sigma:$双射且$\sigma:\bm z\in V\to\bm x\in\mathbb{R}^n,\bm w\in V\to\bm y\in\mathbb{R}^n$\\
\[\sigma(\bm z+\bm w)=\sigma(\bm z)+\sigma(\bm w)\]
\[\sigma(\alpha\bm z)=\alpha\sigma(\bm z)\]
$V$是欧氏空间\\
\[(\sigma\bm z,\sigma\bm w)_{\mathbb{R}}=(\bm z,\bm w)_V\]
设$\bm e_1,\bm e_2...\bm e_n$是$V$中一组标准正交基:\\
$\bm z=\sum_{j=1}^nx_j\bm e_j,\bm w=\sum_{j=1}^ny_j\bm e_j$\\
\[(\bm z,\bm w)=(\sum x_j\bm e_j,\sum y_k\bm e_k)=\sum x_jy_k(\bm e_j,\bm e_k)=\sum_{j=1}^nx_iy_i\]
所以$(\bm z,\bm w)_V=(\bm x,\bm y)_{\mathbb{R}}=(\sigma\bm z,\sigma\bm w)_{\mathbb{R}}$\\
\subsection{\shit{定理:任一$n$为欧氏空间与$\mathbb{R}^n$同构}}
\subsection{\shit{推论1:两个维数相同的欧氏空间同构}}
\subsection{\shit{推论2:正交矩阵$U$决定一个$\mathbb{R}^n$到$\mathbb{R}^n$的同构}}
\[(Ux,Uy)=(x,y)\]
\section{\shit{正交子空间}}
$V_1,V_2$是$V$的线性子空间.若对任意$\bm v_1\in V_1,\bm v_2\in V_2$有$\bm v_1\perp\bm v_2$,称$V_1$与$V_2$正交,记作$V_1\perp V_2$\\
\subsection{\shit{性质(直和??)}}
若$V_1\perp V_2$,则$V_1+V_2=V_1\oplus V_2$\\
$V_1+V_2=\{\bm v_1+\bm v_2|\bm v_1\in V_1,\bm v_2\in V_2\}$\\
$V_1\oplus V_2:\bm v_1\in V_1,\bm v_2\in V_2$,若$\bm v_1+\bm v_2=\bm0$,则$\bm v_1=\bm0,\bm v_2=\bm0$\\
证明:\\
设$\bm v_1\in V_1,\bm v_2\in V_2,\bm v_1+\bm v_2=\bm0$\\
则(内积????)$0=(\bm v_1+\bm v_2,\bm v_1+\bm v_2)=(\bm v_1,\bm v_1)+2(\bm v_1,\bm v_2)+(\bm v_2,\bm v_2)=|\bm v_1|^2+|\bm v_2|^2$\\
所以$\bm v_1=\bm0,\bm v_2=\bm0$\\
\subsection{\shit{正交补}}
若$\bm V_1\perp\bm V_2$且$\bm V_1+\bm V_2=V$,称$V_2(V_1)$是$V_1(V_2)$的正交补,记作$V_2=V_1^\perp$,即设$W$是$V$的子空间,线性空间$W^\perp$满足:\\
\[W^\perp\perp W,W+W^\perp=V\]
\subsection{\shit{正交分解}}
$W$是欧氏空间$V$的一个线性子空间,则$W^\perp$存在唯一,即$V=W\oplus W^\perp$\\
$V:n$维,$W:m$维,$m\le n$\\
$W=span\{\bm w_1,\bm w_2...\bm w_n\}=span\{\bm\eta_1,\bm\eta_2...\bm\eta_n\}$(...正交化,标准正交系)\\
由基扩张定理(???),存在$n-m$个线性无关向量$\bm z_1,\bm z_2...\bm z_{m-n}$\\
使得$\bm\eta_1,\bm\eta_2...\bm\eta_m\bm z_1,\bm z_2...\bm z_{n-m}$是$V$的一组基\\
利用(...正交化方法)得到一组标准正交基$\bm\eta_1,\bm\eta_2...\bm\eta_m,\bm e_1,\bm e_2...\bm e_{n-m}$\\
令$W^{\perp}=span\{\bm e_1,\bm e_2...\bm e_{n-m}\}$\\
\subsubsection{\shit{唯一性}}
设$W_1,W_2$都是$W$的正交补,$V=W\oplus W_1=W\oplus W_2$\\
$\forall w_1\in W_1$,则$w_1\in V=W\oplus W_2$,所以$w_1=w+w_2(w\in W,w_2\in W_2)$\\
但$W_1\perp W$,$(w_1,w)=(w,w)+(w_2,w)=|w|^2+0$\\
所以$w=\bm0,w_1=w_2\in W_2$,$W_1\subset W_2$,反之...\\
给定子空间$W\subset V$,令(和$W$有关的空间)$W_\perp=\{\bm v\in V|\bm v\perp W\}$\\
则\\
1.$W_\perp$是一个子空间,$\bm w_1,\bm w_2\in W_\perp$,则$\alpha_1\bm w_1+\alpha_2\bm w_2\perp W$\\
2.$W_\perp=W^\perp$\\
\subsection{\shit{正交投影}}
$\exists!(?|)\bm w\in W,\bm u\in W^\perp$使得$\bm v=\bm w+\bm u$,称$\bm w$是$\bm v$在$W$上的正交投影\\
记$\bm w=Proj_W\bm v=P\bm v$,则$P(\bm v_1+\bm v_2)=P\bm v_1+P\bm v_2,P(\alpha\bm v)=\alpha P\bm v$\\
$P_W$称为在$W$子空间上的投影算子\\
\subsection{\shit{$\mathbb{R}$上的投影算子}}
设$W=span\{\bm w_1,\bm w_2...\bm w_n\}(n$维列向量),记$A=(\bm w_1,\bm w_2...\bm w_m)_{n\times m}$($n$行,$m$列)\\
对$\forall\bm b\in\mathbb{R}^n$,欲寻找$\bm w=\sum_{j=1}^mx_j\bm w_j\in W$,使得$\bm b-\bm w\perp W$\\
\[\bm w=A\bm x,\bm x=\begin{bmatrix}
x_1\\
x_2\\
...\\
x_m
\end{bmatrix}\]
\[\bm b-A\bm x\perp W,\bm b-A\bm x\perp\bm w_j,j=1,2...m\]
\[(\bm b-A\bm x,\bm w_j)=0,\bm w_j^T(\bm b-A\bm x)=0\]
\[A^\perp(\bm b-A\bm x)=\bm0,A^TA\bm x=A^T\bm b\]
存在唯一解$\bm x$??,$\bm x=(A^TA)^{-1}A^T\bm b$\\
$\bm w=P_W\bm b=A(A^TA)^{-1}A^T\bm b=P\bm b$\\
其中$P=A(A^TA)^{-1}A^T,A=(\bm w_1,\bm w_2...\bm w_n)$\\
1.$P^2=P,0=P(I-P)$(0矩阵)($P\perp(I-P)$???)\\
2.$P^T=P$\\
\section{\shit{复习}}
\subsection{\shit{$r(AB)\le\min\{r(A),r(B)\}$}}
1.$r(A^TA)=r(A)$\\
证明:\\
考虑$A\bm x=0$和$A^TA\bm x=0$,由齐次方程方程组理论,只需证(充分条件)两者解相同(解空间相同,$\ker(A)=\ker(A^TA)$)\\
若$\bm x\in\ker(A)\Rightarrow\bm x\in\ker(A^TA)$.反之,若$\bm x\in\ker(A^TA)$,即$A^TA\bm x=\bm0$\\
两边同乘以$\bm x^T$得\\
$\bm x^TA^TA\bm x=(A\bm x)^TA\bm x=|A\bm x|^2=0$,所以$A\bm x=0$\\
2.若$A\in\mathbb{R}^{m\times n},\bm b\in\mathbb{R}^n$,则线性方程组$A^TA\bm x=A^T\bm b$一定有解\\
证明:\\
只需证$r(A^TA)=r((A^TA,A^T\bm b))$(内积??)\\
\[r(A^TA)\le r((A^TA,A^T\bm b))=r(A^T(A,\bm b))(????)\le r(A^T)=r(A^TA)\]
\section{\shit{齐次线性方程组Revisited}}
$A\in\mathbb{R}^{m\times n},A\bm x=\bm0$,解空间即为$K(A),A=\begin{bmatrix}
\bm a_1\\
\bm a_2\\
...\\
\bm a_m\\
\end{bmatrix}a_j$为$A$的第$j$个解向量\\
$A\bm x\Leftrightarrow\sum_{k=1}^na_{jk}x_k=0\Leftrightarrow(\bm a_j,\bm x)=0\Leftrightarrow\bm a_j\perp\bm x$\\
$\Leftrightarrow\bm x\perp span\{\bm a_1,\bm a_2...\bm a_m\}\overset{\mbox{???????}}{=}span\{\bm a_1,\bm a_2...\bm a_r\}(r=r(A))=V_r$\\
令$\bm e_1,\bm e_2...\bm e_r$是由$\bm a_1,\bm a_2...\bm a_r$经正交化得到的标准正交系,再将其扩充为一组$\mathbb{R}^n$中的标准正交基,则\\
\[\ker(A)=span\{\bm e_{r+1},\bm x_{r+2}...\bm e_n\}=V_r^\perp\]
证明:\\
有前面讨论知$\ker(A)\subset V_r^\perp$,反之,设$\bm x\in V_r^\perp$即\\
\[\bm x=\sum_{j=r+1}^nx_j\bm e_j\]
显然$\bm x\perp\bm e_i,i=1,2...r,(\bm x,\bm e_1)=\sum_{j=r+1}^nx_j(\bm e_j,e_1)=0$\\
所以$\bm x\perp\bm a_j,j=1,2...r$,因为$\bm a_{r+1},\bm a_{r+2}...\bm a_{m}$是$\bm a_1,\bm a_2...\bm a_{r}$的线性组合\\
所以$\bm x\perp\bm a_j,j=r+1,r+2...m$\\
所以$\ker(A)=V_r^\perp\Rightarrow\dim(\ker(A))=n-r(A)$\\
注意到$(\bm x,A\bm y)=(A^T\bm x,\bm y)$\\
\[R(\bm A)=\{\bm{Ax}|\bm x\in V\}\]
(值域???)\\
那么就有$\ker(A)=R(A^T)^\perp$即$\ker(A)\oplus R(A^T)=\mathbb{R}^n$\\
\subsection{\shit{引理:$\dim(R(A))=r(A)$}}
由此$n=r(A)+\dim(\ker(A))$\\
引理的证明:\\
$R(A)=\{A\bm x|\bm x\in\mathbb{R}^n\}$,$A=(\bm\alpha_1,\bm\alpha_2...\bm\alpha_n)$\\
不妨设$\bm\alpha_1,\bm\alpha_2...\bm\alpha_r$线性无关,$r=r(A)$\\
故$\bm\alpha_{r+1},\bm\alpha_{r+2}...\bm\alpha_{m}$可由$\bm\alpha_1,\bm\alpha_2...\bm\alpha_r$线性表示\\
$\forall\bm y\in R(A)$,即$\bm y=A\bm x(\bm x\in\mathbb{R}^n)=\sum_{j=1}^nx_j\bm\alpha_j=\sum_{j=1}^ry_j\bm\alpha_j$\\
\section{\shit{实对称矩阵的对角化}}
\subsection{\shit{实二次型??}}
二维的实二次型??\\
$ax^2+abxy+cy^2$\\
\[[x,y]\begin{bmatrix}
a&b\\
b&c
\end{bmatrix}\begin{bmatrix}
x\\
y
\end{bmatrix}\]\\
三维的实二次型:\\
\[[x_1,x_2,x_3]\begin{bmatrix}
a_{11}&a_{12}&a_{13}\\
a_{12}&a_{22}&a_{23}\\
a_{13}&a_{23}&a_{33}\\
\end{bmatrix}\begin{bmatrix}
x_1\\
x_2\\
x_3
\end{bmatrix}\]
\subsection{\shit{性质1:是对称阵的特征值是实的}}
\[A\bm\xi=\lambda\bm\xi\]
\[\bar{\bm\xi}^TA\bm\xi=\bm\xi^TA\bar{\bm\xi}=\bm\xi^T\bar{\lambda}\bm\xi=\bar\lambda|\bm\xi|^2\]
\subsection{\shit{性质2:实对称阵不同特征值的特征向量正交$\lambda\not=\mu,V_\lambda\perp V_\mu$}}
$A\bm\xi=\lambda\bm\xi,A\bm\eta=\mu\bm\eta$\\
$(\bm\xi,A^T\bm\eta)=(A\bm\xi,\bm\eta)=\lambda(\bm\xi,\bm\eta)=(\bm\xi,A\bm\eta=(A\bm\eta,\bm\xi))=\mu(\bm\eta,\bm\xi)\overset{\lambda\not=\mu}{\Rightarrow}(\bm\xi,\bm\eta)=0$\\
\end{document}