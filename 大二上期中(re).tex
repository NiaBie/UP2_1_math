\documentclass[11pt, a4paper, UTF8]{ctexart}
\usepackage{amsmath}
\usepackage{bm}
\usepackage{multirow}
\usepackage{esint}
\usepackage{geometry}
\usepackage{amssymb}
\pagestyle{empty}
\definecolor{brightpink}{rgb}{1.0, 0.0, 0.5}
\definecolor{ao}{rgb}{0.0, 0.5, 0.0}
\usepackage[colorlinks,linkcolor=ao,anchorcolor=brightpink,citecolor=green,CJKbookmarks=True]{hyperref}
\CTEXsetup[format={\Large\bfseries}]{section}
\geometry{a4paper,top=1cm,bottom=1cm,left=2cm}
%\renewcommand{\baselinestretch}{1}\normalsize
\begin{document}
\indent\\
\large
\raggedright
\def\fuck{\maltese}
\hypertarget{catalog}{}
\tableofcontents
\setcounter{secnumdepth}{-1}
\section{预备知识\protect\hyperlink{catalog}{$\fuck$}}
\subsection{积化和差}
\[\sin\alpha\cos\beta=\dfrac{1}{2}[\sin(\alpha+\beta)+\sin(\alpha-\beta)]\]
$$e^{xi}=\cos x+i\sin x$$
$$e^{-xi}=\cos x-i\sin x$$
$$\cos y=\dfrac{1}{2}(e^{iy}+e^{-iy}),\sin y=\dfrac{-i}{2}(e^{iy}-e^{-iy})$$
\subsection{矩阵求导}
标量/行向量\\
\[\dfrac{\partial y}{\partial\bm x}=
\begin{bmatrix}
\dfrac{\partial y}{\partial x_1}&\dfrac{\partial y}{\partial x_2}&\cdots&\dfrac{\partial y}{\partial x_n}
\end{bmatrix}
\]
标量/列向量\\
\[\dfrac{\partial y}{\partial\bm x}=
\begin{bmatrix}
\dfrac{\partial y}{\partial x_1}\\
\dfrac{\partial y}{\partial x_2}\\
\vdots\\
\dfrac{\partial y}{\partial x_n}
\end{bmatrix}
\]
列向量/行向量\\
\[\dfrac{\partial\bm y}{\partial\bm x}=
\begin{bmatrix}
\dfrac{\partial y_1}{\partial x_1}&\dfrac{\partial y_1}{\partial x_2}&\cdots&\dfrac{\partial y_1}{\partial x_n}\\
\dfrac{\partial y_2}{\partial x_1}&\dfrac{\partial y_2}{\partial x_2}&\cdots&\dfrac{\partial y_2}{\partial x_n}\\
\vdots&\vdots&\vdots&\vdots\\
\dfrac{\partial y_n}{\partial x_1}&\dfrac{\partial y_n}{\partial x_2}&\cdots&\dfrac{\partial y_n}{\partial x_n}\\
\end{bmatrix}
\]
行向量/列向量\\
\[\dfrac{\partial\bm y}{\partial\bm x}=
\begin{bmatrix}
\dfrac{\partial y_1}{\partial x_1}&\dfrac{\partial y_2}{\partial x_1}&\cdots&\dfrac{\partial y_n}{\partial x_1}\\
\dfrac{\partial y_1}{\partial x_2}&\dfrac{\partial y_2}{\partial x_2}&\cdots&\dfrac{\partial y_n}{\partial x_2}\\
\vdots&\vdots&\vdots&\vdots\\
\dfrac{\partial y_1}{\partial x_n}&\dfrac{\partial y_2}{\partial x_n}&\cdots&\dfrac{\partial y_n}{\partial x_n}\\
\end{bmatrix}
\]
矩阵/标量($p\times q$)(原矩阵)\\
\[\dfrac{\partial\bm Y}{\partial x}=
\begin{bmatrix}
\dfrac{\partial y_{11}}{\partial x}&\dfrac{\partial y_{12}}{\partial x}&\cdots&\dfrac{\partial y_{1n}}{\partial x}\\
\dfrac{\partial y_{21}}{\partial x}&\dfrac{\partial y_{22}}{\partial x}&\cdots&\dfrac{\partial y_{2n}}{\partial x}\\
\vdots&\vdots&\vdots&\vdots\\
\dfrac{\partial y_{m1}}{\partial x}&\dfrac{\partial y_{m2}}{\partial x}&\cdots&\dfrac{\partial y_{mn}}{\partial x}\\
\end{bmatrix}
\]
标量/矩阵(行数$\times$列数)($n\times m$)(转置矩阵)\\
\[\dfrac{\partial y}{\partial\bm X}=
\begin{bmatrix}
\dfrac{\partial y}{\partial x_{11}}&\dfrac{\partial y}{\partial x_{21}}&\cdots&\dfrac{\partial y}{\partial x_{p1}}\\
\dfrac{\partial y}{\partial x_{12}}&\dfrac{\partial y}{\partial x_{22}}&\cdots&\dfrac{\partial y}{\partial x_{p2}}\\
\vdots&\vdots&\vdots&\vdots\\
\dfrac{\partial y}{\partial x_{1q}}&\dfrac{\partial y}{\partial x_{2q}}&\cdots&\dfrac{\partial y}{\partial x_{pq}}\\
\end{bmatrix}
\]
\subsection{行列式求导}
\[F_n'(x)
\begin{vmatrix}
f_{11}(x)&f_{12}(x)&\cdots&f_{1n}(x)\\
\vdots&\vdots&\vdots&\vdots\\
f_{i1}(x)&f_{i2}(x)&\cdots&f_{in}(x)\\
\vdots&\vdots&\vdots&\vdots\\
f_{n1}(x)&f_{n2}(x)&\cdots&f_{nn}(x)\\
\end{vmatrix}
'=\sum_{i=1}^n
\begin{vmatrix}
f_{11}(x)&f_{12}(x)&\cdots&f_{1n}(x)\\
\vdots&\vdots&\vdots&\vdots\\
f'_{i1}(x)&f'_{i2}(x)&\cdots&f'_{in}(x)\\
\vdots&\vdots&\vdots&\vdots\\
f_{n1}(x)&f_{n2}(x)&\cdots&f_{nn}(x)\\
\end{vmatrix}
\]
\section{一阶(变系数)线性微分方程\protect\hyperlink{catalog}{$\fuck$}}
$$y'+p(x)y=q(x)$$
\section{二阶常系数线性微分方程\protect\hyperlink{catalog}{$\fuck$}}
$$y''+py'+qy=f(x)$$
\subsection{特征根法}
$$\lambda^2+p\lambda+q=0$$
(1)$p^2-4q>0$,特征方程有两个不同的根$\lambda_1,\lambda_2$,通解为\\
$$y=C_1e^{\lambda_1x}+C_2e^{\lambda_2x}$$
(2)$p^2-4q=0$,特征方程有一个重根$\lambda=-\dfrac{p}{2}$,通解为\\
$$y=C_1e^{\lambda x}+C_2xe^{\lambda x}$$
(3)$p^2-4q<0,\lambda_1=\alpha+\beta i,\lambda_2=\alpha-\beta i$(共轭复根),通解为\\
$$y=e^{\alpha x}(C_1\cos\beta x+C_2\sin\beta x)$$
\section{线性方程组\protect\hyperlink{catalog}{$\fuck$}}
$$\dfrac{d\bm{y}}{dx}=A\bm{y}$$
\section{常数变易法(p44)\protect\hyperlink{catalog}{$\fuck$}}
非齐次方程通解\\
$$y=C_1(x)y(x)+C_2(x)y(x)$$
补充方程\\
$$C_1'(x)y_1(x)+C_2'(x)y_2(x)=0$$
则\\
$$
\begin{cases}
C_1'y_1+C_2'y_2=0\\
C_1'y_1'+C_2'y_2'=f(x)
\end{cases}
$$
\section{刘维尔(Liouville)定理\protect\hyperlink{catalog}{$\fuck$}}
\subsection{刘维尔公式(中国科技大学)}
函数$y_1(x)$与$y_2(x)$是二阶齐次线性微分方程\\
\[y''+p(x)y'+q(x)y=0\]
在区间$(a,b)$上的两个解,则其朗斯基行列式为\\
\[W(x)=W(x_0)e^{-\int_{x_0}^xp(t)dt}\]
其中$x_0$是$(a,b)$上的任意固定点\\
Liouville公式表明,$y_1(x),y_2(x)$的朗斯基行列式要么恒为$0$,要么不等于$0$\\
证明:\\
\[W'(x)=(y_1y'_2-y'_1y_2)'=y_1y''_2+y'_1y'_2-y'_1y'_2-y''_1y_2\]
\[=-y_1(p(x)y'_2+q(x)y_2)+y_2(p(x)y'_1+q(x)y_1)\]
\[=p(x)(-y_1y'_2+y'_1y_2)=-p(x)W(x)\]
则有???\\
\[\int_{x_0}^x\dfrac{dW}{W}=-\int_{x_0}^xp(x)dx\]
\[\ln\dfrac{W(x)}{W(x_0)}=-\int_{x_0}^xp(x)dx\]
\[W(x)=W(x_0)e^{-\int_{x_0}^xp(x)dx}\]
\subsection{刘维尔公式(丁同仁p162)}
\[W(x)=W(x_0)e^{\int_{x_0}^xtr[\bm A(x)]dx}\]
注:\\
\[W(x)=
\begin{vmatrix}
y_{11}(x)&y_{12}(x)&\cdots&y_{1n}(x)\\
y_{21}(x)&y_{22}(x)&\cdots&y_{2n}(x)\\
\vdots&\vdots&\vdots&\vdots\\
y_{n1}(x)&y_{n2}(x)&\cdots&y_{nn}(x)\\
\end{vmatrix}\]
证明:\\
\[\dfrac{dW}{dx}=\sum_{i=1}^n
\begin{vmatrix}
y_{11}&y_{12}&\cdots&y_{1n}\\
\vdots&\vdots&\vdots&\vdots\\
\dfrac{dy_{i1}}{dx}&\dfrac{dy_{i2}}{dx}&\cdots&\dfrac{dy_{in}}{dx}\\
\vdots&\vdots&\vdots&\vdots\\
y_{n1}&y_{n2}&\cdots&y_{nn}\\
\end{vmatrix}
\overset{\mbox{\tiny{右式矩阵相乘}}}{=}\sum_{i=1}^n
\begin{vmatrix}
y_{11}&y_{12}&\cdots&y_{1n}\\
\vdots&\vdots&\vdots&\vdots\\
\sum_{j=1}^na_{ij}y_{j1}&\sum_{j=1}^na_{ij}y_{j2}&\cdots&\sum_{j=1}^na_{ij}y_{jn}\\
\vdots&\vdots&\vdots&\vdots\\
y_{n1}&y_{n2}&\cdots&y_{nn}\\
\end{vmatrix}\]
\[\overset{\mbox{\tiny{行成比例=0}}}{=}\sum_{i=1}^na_{ii}W=tr[\bm A(x)]W\]
%\textbf{刘维尔定理(中国科技大学)}\\
%函数$y_1(x)$与$y_2(x)$是二阶齐次线性微分方程
%$$y''+p(x)y'+q(x)y=0$$
%在区间$(a,b)$上的两个解,则$y_1(x)$与$y_2(x)$在区间$(a,b)$上线性无关的充要条件是朗斯基行列式处处不为$0$\\
%$$W(x)=W(x_0)e^{\int_{x_0}^xtr[A(x)]dx}(a<x<b)$$
\section{傅里叶级数展开\protect\hyperlink{catalog}{$\fuck$}}
\subsection{傅里叶系数}
$$a_k=\dfrac{1}{\pi}\int_{-\pi}^{\pi}f(x)\cos kxdx$$
$$b_k=\dfrac{1}{\pi}\int_{-\pi}^{\pi}f(x)\sin kxdx$$
形式和\\
$$\dfrac{a_0}{2}+\sum_{k=1}^{\infty}(a_k\cos kx+b_k\sin kx)$$
称为$f$的傅里叶级数或傅里叶展开,记为\\
$$f(x)\sim\dfrac{a_0}{2}+\sum_{k=1}^{\infty}(a_k\cos kx+b_k\sin kx)$$
(或)\\
$$\dfrac{a_0}{2}+\sum_{n=1}^{+\infty}(a_n\cos n\omega t+b_n\sin n\omega t)$$
其中$\omega=\dfrac{2\pi}{T}$\\
\subsection{一般区间上的傅里叶级数展开}
(1)$y=f(x),x\in[-l,l]$\\
(2)$y=f(x),x\in[0,l]$\\
(3)$y=f(x),x\in[a,b]$\\
\section{逐项积分、求导(p61)\protect\hyperlink{catalog}{$\fuck$}}
\subsection{逐项积分}
设$f(x)$在$[-\pi,\pi]$上黎曼可积,则\\
\[\dfrac{a_0}{2}x+\sum_{k=1}^{\infty}(\dfrac{a_k}{k}\sin kx+\dfrac{b_k}{k}(1-\cos kx))\]
恒收敛,且\\
\[\int_0^xf(t)dt=\dfrac{a_0}{2}x+\sum_{k=1}^{\infty}(\dfrac{a_k}{k}\sin kx+\dfrac{b_k}{k}(1-\cos kx)),x\in[-\pi,\pi]\]
$$F(x)-\dfrac{a_0}{2}x=\dfrac{A_0}{2}+\sum_{n=1}^{\infty}(A_n\cos nx+B_n\sin nx)$$
其中\\
$$\dfrac{A_0}{2}=\sum_{n=1}^{\infty}\dfrac{b_n}{n},A_n=-\dfrac{b_n}{n},B_n=\dfrac{a_n}{n}$$
\subsection{逐项求导}
设$f(x)$在$[-\pi,\pi]$上连续,$f'(x)$除有限个点外在$[-\pi,\pi]$上处处存在且正常可积,以及$f(\pi)=f(-\pi)$,则$f'(x)$有傅里叶级数\\
$$\dfrac{f'(x+)+f'(x-)}{2}=f'(x)\sim\sum_{n=1}^{\infty}(nb_n\cos nx-na_n\sin nx)$$
\section{Parseval等式(p310)\protect\hyperlink{catalog}{$\fuck$}}
\subsection{Parseval等式}
设$f\in\mathbb{R}^2[-\pi,\pi]$,且f的傅里叶展开为\\
$$f\sim\dfrac{a_0}{2}+\sum_{n=1}^\infty(a_n\cos nx+b_n\sin nx)$$
则\\
$$\dfrac{1}{\pi}\int_{-\pi}^{\pi}f^2(x)dx=\dfrac{a_0^2}{2}+\sum_{n=1}^{\infty}(a_n^2+b_n^2)$$
\subsection{广义Parseval等式}
设$f,g\in\mathbb{R}^2[-\pi,\pi]$,则\\
$$\dfrac{1}{\pi}\int_{-\pi}^{\pi}f(x)g(x)dx=\dfrac{a_0\alpha_0}{2}+\sum_{n=1}^{\infty}(a_n\alpha_n+b_n\beta_n)$$
\section{狄利克雷核(p285)\protect\hyperlink{catalog}{$\fuck$}}
$$\sigma_n(x)=\dfrac{1}{2}\sum_{k=-n}^{n}e^{ikx}=\dfrac{1}{2}+\cos x+\cos 2x+\cdots+\cos nx$$
利用等式\\
$$\sin\dfrac{1}{2}x\cos kx=\dfrac{1}{2}[\sin(k+\dfrac{1}{2})x-\sin(k-\dfrac{1}{2})x]$$
$$\sigma_n(x)=\dfrac{\sin(n+\dfrac{1}{2})}{2\sin(\dfrac{1}{2}x)}$$
\section{黎曼-勒贝格引理(p283)\protect\hyperlink{catalog}{$\fuck$}}
设$f(x)$在$[a,b)$上黎曼可积(或广义绝对可积),$a>0$,则\\
$$\lim\limits_{\lambda\to\infty}\int\limits_a^bf(x)\cos\lambda xdx=0$$
$$\lim\limits_{\lambda\to\infty}\int\limits_a^bf(x)\sin\lambda xdx=0$$
\section{初值问题(p229)\protect\hyperlink{catalog}{$\fuck$}}
\end{document}